\documentclass[10pt,twocolumn]{article}
\usepackage[utf8]{inputenc}
\usepackage[margin=1in,columnsep=0.25in]{geometry}
\usepackage{multicol}
\usepackage{fancyhdr}
\usepackage{graphicx}
\usepackage{float}
\usepackage{wrapfig}
\usepackage{lipsum}
\usepackage{changepage}

\pagestyle{fancy}
\fancyhf{}
\fancyhead[LE,RO]{\thepage}
\fancyhead[LO,RE]{Layout Test Document}
\fancyfoot[C]{Complex Layout Testing}

\title{Advanced Layout Testing Document}
\author{DVI Tools Layout Analysis}
\date{\today}

\begin{document}

\maketitle

\section{Two-Column Layout}

This document is set in two-column mode to test complex page layout parsing. The DVI file will contain intricate positioning commands to handle the column breaks and text flow.

\lipsum[1]

\subsection{Column Break Control}

Text before manual column break.

\columnbreak

Text after manual column break in the second column.

\lipsum[2]

\section{Multi-Column Environments}

\begin{multicols}{3}
\subsection{Three-Column Section}
This section uses the multicols environment to create three columns. Each column should be narrow and contain flowing text.

Column 1 content with various text formatting. \textbf{Bold text} and \textit{italic text} within the narrow columns.

Column continues with more content to fill the space and demonstrate text flow across multiple narrow columns.

\columnbreak

Middle column starts here with different content. The positioning of text in the middle column requires precise DVI commands.

Mathematical expressions in narrow columns: $E = mc^2$ and $\pi \approx 3.14159$.

\columnbreak

Third column contains the final portion of the multi-column text. This tests the DVI parser's ability to handle complex column layouts.

Final content in the rightmost column with special characters: \&, \%, \$, \#, \_, \{, \}.
\end{multicols}

\section{Floating Elements}

\subsection{Figure Placement}

\begin{figure}[h]
\centering
\rule{5cm}{3cm}
\caption{A placeholder figure in two-column layout}
\label{fig:placeholder}
\end{figure}

Figure~\ref{fig:placeholder} demonstrates floating element positioning in a two-column document. The DVI commands for figure placement are complex.

\lipsum[3]

\subsection{Wrapped Text Around Graphics}

\begin{wrapfigure}{r}{0.3\linewidth}
\centering
\rule{2cm}{2cm}
\caption{Wrapped figure}
\label{fig:wrapped}
\end{wrapfigure}

This paragraph contains wrapped text around Figure~\ref{fig:wrapped}. The wrapfig package creates complex positioning that requires sophisticated DVI commands for proper text flow.

The text continues to wrap around the figure, demonstrating how the DVI format handles non-rectangular text areas and complex geometric layouts.

More wrapped text to fill the space and show the complete wrapping behavior around the floating graphic element.

\lipsum[4]

\section{Indentation and Spacing Variations}

\subsection{Paragraph Indentation}

Normal paragraph with standard indentation. The first line is indented according to the document class settings.

\noindent This paragraph has no indentation due to the noindent command. The DVI positioning reflects this spacing change.

\begin{quote}
This is a quoted paragraph with different margin settings. The left and right margins are increased, creating a block quote effect in the DVI layout.
\end{quote}

\begin{quotation}
This quotation environment creates different spacing and indentation patterns compared to the quote environment. The DVI file will show these layout differences.
\end{quotation}

\subsection{Custom Indentation}

\begin{adjustwidth}{2cm}{1cm}
This text block uses the adjustwidth environment to create custom left and right margins. The left margin is increased by 2cm and the right margin by 1cm.

This creates asymmetric layout that tests the DVI parser's handling of non-standard text positioning and margin calculations.
\end{adjustwidth}

\section{List Layouts in Columns}

\subsection{Itemized Lists}

\begin{itemize}
\item First item in two-column layout
\item Second item with longer text that may wrap within the narrow column width
\item Third item with mathematical content: $\sum_{i=1}^n i = \frac{n(n+1)}{2}$
\item Fourth item testing special characters: \textbf{bold}, \textit{italic}, \texttt{mono}
\end{itemize}

\subsection{Enumerated Lists}

\begin{enumerate}
\item Numbered item one
\item Numbered item two with extended content that spans multiple lines within the column constraints
\item Mathematical numbered item: $\int_0^1 x^2 dx = \frac{1}{3}$
\item Final numbered item
\end{enumerate}

\section{Table Layouts}

\subsection{Simple Table in Columns}

\begin{table}[h]
\centering
\begin{tabular}{|c|c|c|}
\hline
A & B & C \\
\hline
1 & 2 & 3 \\
4 & 5 & 6 \\
\hline
\end{tabular}
\caption{Simple table in two-column layout}
\end{table}

\subsection{Wide Table Spanning Columns}

\begin{table*}[t]
\centering
\begin{tabular}{|l|c|r|p{3cm}|c|}
\hline
Left & Center & Right & Paragraph & Math \\
\hline
Item 1 & 100 & \$50.00 & Long text content & $x^2$ \\
Item 2 & 200 & \$75.50 & More paragraph text & $y^3$ \\
Item 3 & 150 & \$62.25 & Final description & $z^4$ \\
\hline
\end{tabular}
\caption{Wide table spanning both columns}
\end{table*}

The table above spans both columns using the table* environment, creating complex layout positioning in the DVI file.

\section{Page Layout Testing}

\subsection{Manual Spacing}

Text with manual spacing controls:

\vspace{0.5cm}
Vertical space above this line.

Text with horizontal spacing: \hspace{1cm} indented portion.

\vspace{1cm}
Larger vertical space to test DVI spacing commands.

\subsection{Line Spacing Variations}

{\baselineskip=1.5\baselineskip
This paragraph has increased line spacing using baselineskip modification. The lines are spaced 1.5 times the normal distance.

Multiple lines of text with increased spacing to demonstrate the effect on DVI positioning commands.
}

{\baselineskip=0.8\baselineskip
This paragraph has decreased line spacing. The lines are closer together than normal.

Compressed text layout for DVI analysis.
}

\section{Complex Positioning}

\subsection{Overlapping and Absolute Positioning}

Normal text flow continues here.

\hbox to 0pt{%
\hspace{-2cm}%
\vbox{%
\hsize=4cm
\textbf{Overlaid text box} that appears to the left of the normal text flow.
This tests complex positioning in DVI.
}%
\hss
}

The normal text continues after the overlaid box, demonstrating how DVI handles overlapping content and absolute positioning.

\lipsum[5]

\subsection{Zero-Width and Zero-Height Boxes}

Text before zero-width content.%
\hbox to 0pt{ZERO WIDTH\hss}%
Text after zero-width insertion.

\vbox to 0pt{%
ZERO HEIGHT BOX
\vss
}%

Text continues after zero-height box insertion.

\section{Font Size Mixing in Layout}

{\tiny Tiny text} mixed with {\normalsize normal text} and {\Large large text} in the same line creates complex vertical alignment challenges for DVI parsing.

\begin{center}
{\Huge HUGE CENTERED}\\
{\large with large subtitle}\\
{\small and small details}
\end{center}

Mixed sizing within paragraphs: Start with normal text, then {\scriptsize switch to script size for some words}, back to normal, then {\LARGE LARGE for emphasis}, and finally {\footnotesize footnote size to end}.

\section{Final Layout Tests}

\onecolumn

\section{Single Column Return}

This section returns to single-column layout to test the transition from two-column to one-column within the same document.

\lipsum[6-7]

\begin{multicols}{4}
Final four-column section to test narrow column handling and complex text flow patterns that create intricate DVI positioning commands.

Column 1 of 4 with tightly packed text and minimal spacing.

\columnbreak

Column 2 with mathematical content: $\sum_{n=1}^{\infty} \frac{1}{n^2} = \frac{\pi^2}{6}$.

\columnbreak

Column 3 with special formatting: \textbf{bold}, \textit{italic}, \texttt{monospace}.

\columnbreak

Final column 4 with mixed content and complex character positioning.
\end{multicols}

\end{document}