\documentclass[12pt]{article}
\usepackage[utf8]{inputenc}
\usepackage[T1]{fontenc}
\usepackage{babel}
\usepackage{amsmath}
\usepackage{textcomp}

\title{Multilingual Document for DVI Testing}
\author{International Character Test}
\date{\today}

\begin{document}

\maketitle

\section{Latin Characters and Accents}

\subsection{Basic Accents}
Acute: \'a \'e \'i \'o \'u \'y
Grave: \`a \`e \`i \`o \`u
Circumflex: \^a \^e \^i \^o \^u
Umlaut: \"a \"e \"i \"o \"u \"y
Tilde: \~a \~n \~o
Cedilla: \c{c} \c{s}

\subsection{European Languages}
French: caf\'e, na\"ive, r\^ole, \'ecole, fran\c{c}ais
German: M\"unchen, Stra\ss{}e, \"Osterreich, wei\ss{}
Spanish: ni\~no, se\~nor, cami\'on, coraz\'on
Italian: perch\'e, citt\`a, pi\`u, universit\`a
Portuguese: portugu\^es, cora\c{c}\~ao, informa\c{c}\~ao

\section{Special Characters and Symbols}

\subsection{Currency and Commercial}
Currency: \$ \pounds{} \textyen{} \texteuro{} \textcent{}
Commercial: \textcopyright{} \textregistered{} \texttrademark{} \textparagraph{} \textsection{}

\subsection{Mathematical Symbols in Text}
Degrees: 90\textdegree{} angle
Plus/minus: $\pm$ 3.5\%
Multiplication: 5 $\times$ 7 = 35
Division: 20 $\div$ 4 = 5
Fractions: \textonehalf{} \textonequarter{} \textthreequarters{}

\subsection{Punctuation and Quotes}
English quotes: ``Hello world''
German quotes: ,,Guten Tag``
French quotes: <<Bonjour>>
Dashes: en-dash (-), em-dash (---), hyphen (-)
Ellipsis: \ldots{}

\section{Font Variations and Emphasis}

\subsection{Text Styles}
\textbf{Bold text} with various accents: \textbf{caf\'e}, \textbf{M\"unchen}
\textit{Italic text} with special chars: \textit{\textcopyright{} 2025}
\texttt{Monospace text} with symbols: \texttt{\$ \# \& \%}
\textsf{Sans-serif text} with accents: \textsf{\'ecole fran\c{c}aise}

\subsection{Combined Styles}
\textbf{\textit{Bold italic}} with \textbf{\textit{caf\'e}}
\texttt{\textbf{Bold monospace}} with \texttt{\textbf{\$ 123.45}}
\textsf{\textit{Italic sans}} with \textsf{\textit{na\"ive}}

\section{Size Variations with Special Characters}

{\tiny Tiny: caf\'e, M\"unchen, ni\~no}
{\scriptsize Script: \textcopyright{} \pounds{} \texteuro{}}
{\footnotesize Footnote: <<Bonjour>> ,,Guten Tag``}
{\small Small: \textonehalf{} \textonequarter{} \textthreequarters{}}
{\large Large: \'ecole fran\c{c}aise}
{\Large Larger: Se\~nor Ni\~no}
{\LARGE LARGE: CITT\`A}
{\huge Huge: STRA\ss{}E}
{\Huge Huge: M\"UNCHEN}

\section{Mathematical Text Integration}

When we write about the caf\'e's revenue of \texteuro{}1,234.56, we might express it mathematically as $R = 1234.56$ euros. The temperature might be 23.5\textdegree{}C, or in the equation $T = 23.5$ degrees Celsius.

A mathematical expression with text: Let $f: \mathbb{R} \to \mathbb{R}$ be defined such that for all \'ecoles, we have:
\begin{equation}
\int_{\text{caf\'e}}^{\text{universit\'e}} f(x) dx = \pi \cdot \text{r\^ole}
\end{equation}

\section{Line Breaking and Hyphenation}

This section tests how different character sets affect line breaking and hyphenation patterns.

Long German compound words: Donaudampfschifffahrtselektrizit\"atenhauptbetriebswerkbauunterbeamtengesellschaft and Kraftfahrzeug-Haftpflichtversicherung.

French with liaison: l'\'ecole, l'universit\'e, aujourd'hui, c'est-\`a-dire.

Spanish with special characters: El ni\~no peque\~no comi\'o en el cami\'on del se\~nor.

Mixed languages in one paragraph: We visited the caf\'e in M\"unchen where the se\~nor from Espa\~na met the professeur from l'universit\'e fran\c{c}aise to discuss the new Stra\ss{}e project.

\section{Special Spacing and Kerning Tests}

\subsection{Character Combinations}
AV WA To Te Yo P. T. f' f' '' ``'' ,,``
VA WA LT Tr Ty We Wo ff fi fl ffi ffl
123 456 789 (123) [456] \{789\}

\subsection{Accent Combinations}
\`a\'a \^a\"a \~a\c{a} \'e\`e \"e\^e \~e\c{e}
Multiple accents create complex character spacing.

\subsection{Mixed Scripts}
Latin: Hello World!
Math: $\alpha + \beta = \gamma$
Symbols: \textcopyright{} \texttrademark{} \textregistered{}
Currency: \$1,234.56 + \texteuro{}987.65 = \pounds{}2,000.00

\section{Table with International Content}

\begin{center}
\begin{tabular}{|l|l|l|l|}
\hline
Language & Greeting & Currency & Special \\
\hline
English & Hello & \$123.45 & ``quotes'' \\
French & Bonjour & \texteuro{}123,45 & <<guillemets>> \\
German & Guten Tag & \texteuro{}123,45 & ,,Anf\"uhrung`` \\
Spanish & Hola & \texteuro{}123,45 & ¿pregunta? \\
\hline
\end{tabular}
\end{center}

\section{Bibliography with International Names}

Some famous scientists with accented names:

Ren\'e Descartes (French philosopher and mathematician)
Andr\'e-Marie Amp\`ere (French physicist)
Anders Jonas \AA{}ngstr\"om (Swedish physicist)
Niels Bohr (Danish physicist)
Henri Poincar\'e (French mathematician)
\'Emile Borel (French mathematician)

\section{Footnotes with Special Characters}

This text has footnotes with international content\footnote{Note in French: Cette note contient des caract\`eres sp\'eciaux comme \'e, \`a, \c{c}.} and mathematical symbols\footnote{Mathematical note: $\int_0^1 x^2 dx = \frac{1}{3}$ where \textonethird{} is the result.}.

German footnote example\footnote{Deutsche Fu\ss{}note: M\"unchen ist eine sch\"one Stadt in \"Osterreich... nein, in Bayern!} with umlauts and eszett.

\end{document}